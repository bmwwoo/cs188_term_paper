%%%%%%%%%%%%%%%%%%%%%%%%%%%%%%%%%%%%%%%%%
% Structured General Purpose Assignment
% LaTeX Template
%
% This template has been downloaded from:
% http://www.latextemplates.com
%
% Original author:
% Ted Pavlic (http://www.tedpavlic.com)
%
% Note:
% The \lipsum[#] commands throughout this template generate dummy text
% to fill the template out. These commands should all be removed when
% writing assignment content.
%
%%%%%%%%%%%%%%%%%%%%%%%%%%%%%%%%%%%%%%%%%

%----------------------------------------------------------------------------------------
%	PACKAGES AND OTHER DOCUMENT CONFIGURATIONS
%----------------------------------------------------------------------------------------

\documentclass{article}

\usepackage{fancyhdr} % Required for custom headers
\usepackage{lastpage} % Required to determine the last page for the footer
\usepackage{extramarks} % Required for headers and footers
\usepackage{graphicx} % Required to insert images
\usepackage{lipsum} % Used for inserting dummy 'Lorem ipsum' text into the template

% Margins
\topmargin=-0.45in
\evensidemargin=0in
\oddsidemargin=0in
\textwidth=6.5in
\textheight=9.0in
\headsep=0.25in

\linespread{1.1} % Line spacing

% Set up the header and footer
\pagestyle{fancy}
\lhead{\hmwkAuthorName} % Top left header
\chead{\hmwkClass\ (\hmwkClassInstructor\ \hmwkClassTime): \hmwkTitle} % Top center header
\rhead{\firstxmark} % Top right header
\lfoot{\lastxmark} % Bottom left footer
\cfoot{} % Bottom center footer
\rfoot{Page\ \thepage\ of\ \pageref{LastPage}} % Bottom right footer
\renewcommand\headrulewidth{0.4pt} % Size of the header rule
\renewcommand\footrulewidth{0.4pt} % Size of the footer rule

\setlength\parindent{0pt} % Removes all indentation from paragraphs

%----------------------------------------------------------------------------------------
%	DOCUMENT STRUCTURE COMMANDS
%	Skip this unless you know what you're doing
%----------------------------------------------------------------------------------------

% Header and footer for when a page split occurs within a problem environment
\newcommand{\enterProblemHeader}[1]{
\nobreak\extramarks{#1}{#1 continued on next page\ldots}\nobreak
\nobreak\extramarks{#1 (continued)}{#1 continued on next page\ldots}\nobreak
}

% Header and footer for when a page split occurs between problem environments
\newcommand{\exitProblemHeader}[1]{
\nobreak\extramarks{#1 (continued)}{#1 continued on next page\ldots}\nobreak
\nobreak\extramarks{#1}{}\nobreak
}

\setcounter{secnumdepth}{0} % Removes default section numbers
\newcounter{homeworkProblemCounter} % Creates a counter to keep track of the number of problems

\newcommand{\homeworkProblemName}{}
\newenvironment{homeworkProblem}[1][Problem \arabic{homeworkProblemCounter}]{ % Makes a new environment called homeworkProblem which takes 1 argument (custom name) but the default is "Problem #"
\stepcounter{homeworkProblemCounter} % Increase counter for number of problems
\renewcommand{\homeworkProblemName}{#1} % Assign \homeworkProblemName the name of the problem
\section{\homeworkProblemName} % Make a section in the document with the custom problem count
\enterProblemHeader{\homeworkProblemName} % Header and footer within the environment
}{
\exitProblemHeader{\homeworkProblemName} % Header and footer after the environment
}

\newcommand{\problemAnswer}[1]{ % Defines the problem answer command with the content as the only argument
\noindent\framebox[\columnwidth][c]{\begin{minipage}{0.98\columnwidth}#1\end{minipage}} % Makes the box around the problem answer and puts the content inside
}

\newcommand{\homeworkSectionName}{}
\newenvironment{homeworkSection}[1]{ % New environment for sections within homework problems, takes 1 argument - the name of the section
\renewcommand{\homeworkSectionName}{#1} % Assign \homeworkSectionName to the name of the section from the environment argument
\subsection{\homeworkSectionName} % Make a subsection with the custom name of the subsection
\enterProblemHeader{\homeworkProblemName\ [\homeworkSectionName]} % Header and footer within the environment
}{
\enterProblemHeader{\homeworkProblemName} % Header and footer after the environment
}

%----------------------------------------------------------------------------------------
%	NAME AND CLASS SECTION
%----------------------------------------------------------------------------------------

\newcommand{\hmwkTitle}{Market Chirp} % Assignment title
% \newcommand{\hmwkDueDate}{Monday,\ January\ 1,\ 2012} % Due date
\newcommand{\hmwkClass}{ABCs} % Course/class
\newcommand{\hmwkClassTime}{10:30am} % Class/lecture time
\newcommand{\hmwkClassInstructor}{Jones} % Teacher/lecturer
\newcommand{\hmwkAuthorName}{Alex Fong, Brandon Woo, Chris Konstad, Sakib Shaikh} % Your name

%----------------------------------------------------------------------------------------
%	TITLE PAGE
%----------------------------------------------------------------------------------------

\title{
\vspace{2in}
\textmd{\textbf{\hmwkClass:\ \hmwkTitle}}\\
% \normalsize\vspace{0.1in}\small{Due\ on\ \hmwkDueDate}\\
\vspace{0.1in}\large{\textit{\hmwkClassInstructor\ \hmwkClassTime}}
\vspace{3in}
}

\author{\textbf{\hmwkAuthorName}}
\date{} % Insert date here if you want it to appear below your name

%----------------------------------------------------------------------------------------

\begin{document}

\maketitle

%----------------------------------------------------------------------------------------
%	TABLE OF CONTENTS
%----------------------------------------------------------------------------------------

%\setcounter{tocdepth}{1} % Uncomment this line if you don't want subsections listed in the ToC

\newpage
\tableofcontents
\newpage

%----------------------------------------------------------------------------------------
%	PROBLEM 1
%----------------------------------------------------------------------------------------

% To have just one problem per page, simply put a \clearpage after each problem

\begin{homeworkProblem}
Market Chirp is a stock recommendation web application developed in the context of UCLA CS 188/219, Scalable Internet Services in Fall 2015 by the team ABCs. The application analyzes tweet sentiments concerning certain stocks and recommends if the stock is bearish or bullish.
\\
\indent Market Chirp analyzes over 2,800 stocks in the New York Stock Exchange. Once a user creates an account, an aggregate of data appears in the dashboard. In the dashboard after selecting a stock, one can view the Twitter feed, the tweet sentiment of the stock, and the stock history. Common stock data such as the market cap and current versus historical stock price is gathered from Yahoo! Finance's API. In addition, Market Chirp takes a mix of the most influential and most recent tweets containing the current stock ticker's symbol of the past twenty-four hours and runs the tweets through a sentiment analyzer. The average of the sentiments is displayed along with a tweet that most closely matches the sentiment.
\\
\indent In addition, one can favorite stocks to keep a running list of stocks to view with ease at later times. The favorited stocks appear in the sidebar and can be unfavorited at anytime.
\\
\indent The goal of Market Chirp is to quickly analyze the public's opinion of high-quality stocks through short notes on Twitter. Since the feed of Tweets is queried in real time when a user searches for a certain stock, the sentiment analysis of a stock is the public's current opinion of is a stock is bullish or bearish. For example, if a stock is on the downhill trend for its market price but its Twitter sentiment is on the uptrend, one may take that as an indication to buy the stock since more people will then value the stock higher. Market Chirp's ultimate goal is not to predict the stock market as the sentiment is based only on the dataset of people tweeting about a stock but is to give insight on the predicted future of stocks based on public opinion.
\\
\indent Market Chirp is implemented in Ruby on Rails (Ruby 2.2.1 and Rails 4.2.4). The backend data store is a relational database management system, in our case MySQL. Since the purpose of CS 188/219: Scalable Internet Services is to build and deploy a scalable web service, this report will therefore discuss the deployment, performance and scalability of Market Chirp.
\vspace{10pt} % Question

\problemAnswer{ % Answer
\begin{center}
\includegraphics[width=0.75\columnwidth]{example_figure} % Example image
\end{center}
Development\\
Through the development of Market Chirp, our team used an Agile framework. Here we had weekly sprint planning meetings to discuss where the project was headed along with retrospective meetings to gain a perspective of what we had accomplished or still had yet to implement over the past week. Tasks not completed by the specified sprint date were automatically moved into a backlog or icebox in order to be completed in future sprints. By using these stand-up meetings, our team was able to precisely figure out the tasks that needed to be finished and how our individual features were pushed into the big picture of the entire product and its scalability.
\\
Pivotal Tracker guided us in following the Agile development framework. With Pivotal Tracker integrated with Github to keep track of certain features or issues, our commits and releases could be synced up to reflect the current status of the application. We were able to assign tickets to certain people in order to split the work up in an efficient manner. By modularizing the tasks, implementing the features and bug fixes was more smooth that using another development framework such as Waterfall Planning. As the quarter moved along however, Pivotal Tracker started being faded out as we were in constant connection with each other still keeping up to date where we should be with the project.
\\
Travis CI was set up as a continuous integration service for building and testing project. We created numerous test cases that would be run each commit to see the version control's reliability. Connecting Travis CI to Github allowed us to keep commiting new code and test its reliability and keep it bug free.
\\
Our version control management system was Git. By keeping the master branch as our production branch, our releases on AWS were always in sync with the latest release on our remote Github. We all worked on separate branches and used pull requests to merge our new code into the other branches eventually merging into master. By constantly rebasing our branches, our development history is clean and looks as though only one person had developed it. Separating the features and bug fixes into their own respective branch allowed each developer to focus on his own feature without conflicting with other developers. After merging in the features, the application's version control was clean.
\\
We also performed pair programming in certain scenarios such as when developing our Memcached feature optimization. Pair programming allowed us to minimize our errors as we wrote our code since we had multiple eyes on the screen at a time. In doing so, our time of outlining a feature branch to pushing production ready code was optimized a substantial amount.
\\
Since Rails is a test driven development framework, it automatically created testing templates for the classes we developed. We made sure to follow test driven development in which one creates test cases for a feature before actually implementing the feature. By following TDD, edge cases were covered as we programmed out the features since we were aware of them beforehand. Since we created ample test cases, the majority of our code was covered and allowed us to be confident our website would not crash when scaled upward.
\\
}
\end{homeworkProblem}

%----------------------------------------------------------------------------------------
%	PROBLEM 2
%----------------------------------------------------------------------------------------

\begin{homeworkProblem}[Exercise \#\arabic{homeworkProblemCounter}] % Custom section title
\lipsum[3] % Question

%--------------------------------------------

\begin{homeworkSection}{(a)} % Section within problem
\lipsum[4]\vspace{10pt} % Question

\problemAnswer{ % Answer
\lipsum[5]
}
\end{homeworkSection}

%--------------------------------------------

\begin{homeworkSection}{(b)} % Section within problem
\problemAnswer{ % Answer
\lipsum[6]
}
\end{homeworkSection}

%--------------------------------------------

\end{homeworkProblem}

%----------------------------------------------------------------------------------------
%	PROBLEM 3
%----------------------------------------------------------------------------------------

\begin{homeworkProblem}[Prob. \Roman{homeworkProblemCounter}] % Roman numerals

%--------------------------------------------

\begin{homeworkSection}{\homeworkProblemName:~(a)} % Using the problem name elsewhere
\problemAnswer{ % Answer
\lipsum[7]
}
\end{homeworkSection}

%--------------------------------------------

\begin{homeworkSection}{\homeworkProblemName:~(b)}
\lipsum[8]\vspace{10pt} % Question

\problemAnswer{ % Answer
\lipsum[9]
}
\end{homeworkSection}

%--------------------------------------------

\end{homeworkProblem}

%----------------------------------------------------------------------------------------
%	PROBLEM 4
%----------------------------------------------------------------------------------------

\begin{homeworkProblem}[Prob. \Roman{homeworkProblemCounter}] % Roman numerals
\problemAnswer{ % Answer
\lipsum[10]
}
\end{homeworkProblem}

%----------------------------------------------------------------------------------------

\end{document}
